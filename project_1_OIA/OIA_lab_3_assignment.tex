\documentclass{article}

\usepackage{hyperref}
\hypersetup{
	colorlinks=true,
	linkcolor=blue,
	urlcolor=cyan,}
\usepackage{booktabs}
\usepackage{textgreek}

\input{../structure.tex} % Include the file specifying the document structure and custom commands

%----------------------------------------------------------------------------------------
%	ASSIGNMENT INFORMATION
%----------------------------------------------------------------------------------------

\title{OIA Lab 3: In Class Assignment}
\author{BIOE 385 Bioinstrumentation Laboratory} 
\date{\textit{Print a copy of this sheet and bring it to lab!}}
%----------------------------------------------------------------------------------------

\begin{document}
\large
\maketitle

\textbf{Student Name:}\hfill 	\textbf{Total Grade:\ \ \ \ /20}\vspace{0.5cm}

\textbf{Student Name:}\hfill 	\textbf{Total Grade:\ \ \ \ /20}\vspace{0.5cm}

\textbf{Student Name:}\hfill 	\textbf{Total Grade:\ \ \ \ /20}\\

\underline{Pre-Lab Assignment}\\

\textbf{Student Name:}\hfill 	\textbf{Grade:\ \ \ \ /5}\vspace{0.5cm}

\textbf{Student Name:}\hfill 	\textbf{Grade:\ \ \ \ /5}\vspace{0.5cm}

\textbf{Student Name:}\hfill 	\textbf{Grade:\ \ \ \ /5}\\

Have the Instructor or Teaching Assistant initial that you have demonstrated successful completion of each task where designated.

\section*{In-Lab Assignment}
Set up a LabView Interface to view and collect the data from your device.\\

To achieve full credit for this task, students must demonstrate a strong effort working in LabView and developing their GUI as well as a working data collection system. If one student appears to be doing all the work, the TA can assign different lab grades for each student.\hfill \textbf{\underline{\hspace{1cm}}/15}\\TA check: \underline{\hspace{2cm}}

\end{document}

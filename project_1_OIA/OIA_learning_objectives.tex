\documentclass{article}

\input{../structure.tex} % Include the file specifying the document structure and custom commands

%----------------------------------------------------------------------------------------
%	ASSIGNMENT INFORMATION
%----------------------------------------------------------------------------------------

\title{Optical Immunoassay Assignment (OIA) Learning Objectives}
\author{BIOE 385 Bioinstrumentation Laboratory} 
\date{}
%----------------------------------------------------------------------------------------

\begin{document}
\large
\maketitle
\section*{Lab 1}
Students should be able to:
\begin{itemize}
	\item Use NI ELVIS function generator to output a variety of waveforms   
	\item Use NI ELVIS variable power supplies as a source to generate DC voltages   
	\item Use NI ELVIS to measure voltages using the oscilloscope and DMM   
	\item Describe how triggers work and use them to stabilize signals   
	\item Map out connections between pins in each of the sections of the protoboard   
	\item Design and build a voltage divider given a desired transfer function   
	\item Design, build and balance a Wheatstone bridge   
	\item Explain how strain gages work   
	\item Design circuits using potentiometers and describe applications where they might be useful   
	\item Explain the time constant and how voltage changes in an RC circuit as a function of time   
\end{itemize}

\section*{Lab 2}
Students should be able to:
\begin{itemize}
	\item Design and build a circuit to control the amount of light emitted from an LED   
	\item Design and build two different circuits that detect different light intensities using a photodetector   
	\item Use operational amplifiers to design circuits   
	\item Calculate the gain and build circuits to amplify voltage using inverting amplifiers   
	\item Explain the difference (and some advantages and disadvantages) between the photovoltaic and photoconductive circuits built in class   
\end{itemize}

\section*{Lab 3}
Students should be able to:
\begin{itemize}
	\item Acquire data from different channels using LabView   
	\item Modify properties of the DAQ to acquire the desired data   
	\item Use Arithmetic and Comparison functions in LabView to manipulate signals   
	\item Create VIs using for-loops and while-loops and describe the main differences between them   
	\item Use waveform graphs and charts to correctly plot the acquired data (display correct labels, units, scaling, etc)   
	\item Use a variety of controls and indicators to create an easy to use VI   
\end{itemize}

\section*{Lab 4}
Students should be able to:
\begin{itemize}
	\item Explain how the circuits built in class can detect nanoshell concentration and antigen presence   
	\item Justify the selection of components used in the circuits designed in lab   
	\item Explain the relation between the LED and FDS 100 and how external factors might affect the functioning of the device by interfering with this setup   
	\item Explain the LabView VI built to determine nanoshell concentration for a given sample   
	\item Identify and describe the main limitations of the devices   
	\item Calculate the accuracy and precision of each circuit and select one for a final recommendation 
\end{itemize}
\end{document}

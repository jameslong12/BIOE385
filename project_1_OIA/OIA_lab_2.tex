\documentclass{article}

\usepackage{hyperref}
\hypersetup{
	colorlinks=true,
	linkcolor=blue,
	urlcolor=cyan,}
\usepackage{booktabs}
\usepackage{textgreek}

\input{../structure.tex} % Include the file specifying the document structure and custom commands

%----------------------------------------------------------------------------------------
%	ASSIGNMENT INFORMATION
%----------------------------------------------------------------------------------------

\title{OIA Lab 2: Basic Circuits for Optical Immunoassay}
\author{BIOE 385 Bioinstrumentation Laboratory} 
\date{}
%----------------------------------------------------------------------------------------

\begin{document}
\large
\maketitle

\section*{Goals}
\begin{enumerate}
	\item Understand the difference between a photoconductive and photovoltaic circuit
	\item Build the two different types of photodetector circuits
\end{enumerate}

\section*{In-lab Assignment}
It is important to use your knowledge of ELVIS and electronics to make your own decisions about the circuits you are assigned to build. When you have completed a task, demonstrate it to the TA or instructor.

\begin{enumerate}
	\item Build a circuit to vary the light emitted from the LED with a potentiometer (see diagram below for a starting point). By adjusting the potentiometer you should be able to vary the amount of light emitted from the LED. Ultimately, you will need to build 2 of these circuits for your two methods of detecting the light.
		\begin{figure}[h]
    	\includegraphics[width=0.6\textwidth]{lab_2_fig_1.png}
    	\centering
		\end{figure}
		
		\begin{warn}
		Apply a voltage that is high enough to drive the LED, but not so high that it will blow out!	
		\end{warn}

	\item Build the circuit depicted in the FDS 100 package insert to detect the light with the photodetector. You will want to position this circuit conveniently to one of the LED circuits you build so that a sample can be placed between them in your future work. Demonstrate that is it able to detect light and varies in output when you block light by placing your finger above the detector. Ideally you will select resistor components that will maximize the range of voltage output between the full light and no light situations. If you are not sure what sizes of resistors to use, select a reasonable size and then adjust from there. You will continue to optimize this in future weeks.
	\item Build the circuit using an OP07 Op Amp to convert current from the photodetector to voltage in order to detect the light from the LED with the photodetector. Demonstrate that is it able to detect light and varies in output when you block light by placing your finger above the detector. Ideally you will select resistor components that will provide a gain to maximize the range of voltage output between the full light and no light situations. You will continue to optimize this in future weeks.
		\begin{figure}[h!]
    	\includegraphics[width=0.8\textwidth]{lab_2_fig_2.png}
    	\centering
		\end{figure}
		
		Since $V_- = V_+ = 0$ (ground), the photocurrent generated by the photodiode flows through the feedback resistor, $R_f$.\\
		
		With feedback resistor $R_f$, the amplifier circuit converts the photodiode current $I_{sc}$ to an output voltage $V_{out} = I_{sc} \times R_f$.	
\end{enumerate}

\begin{info}
	Additional task ideas if you have time (highly recommended):
	\begin{itemize}
		\item Calibrate circuits with LED
		\item Think about/program method to view data collected in LabView
		\item Transition to controlling the LED with LabView instead of a potentiometer
	\end{itemize}
\end{info}

\end{document}

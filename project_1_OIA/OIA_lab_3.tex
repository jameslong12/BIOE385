\documentclass{article}

\usepackage{hyperref}
\hypersetup{
	colorlinks=true,
	linkcolor=blue,
	urlcolor=cyan,}
\usepackage{booktabs}
\usepackage{textgreek}

\input{../structure.tex} % Include the file specifying the document structure and custom commands

%----------------------------------------------------------------------------------------
%	ASSIGNMENT INFORMATION
%----------------------------------------------------------------------------------------

\title{OIA Lab 3: LabView Interface for Optical Immunoassay}
\author{BIOE 385 Bioinstrumentation Laboratory} 
\date{}
%----------------------------------------------------------------------------------------

\begin{document}
\large
\maketitle

\section*{Goals}
\begin{enumerate}
	\item Refresh your understanding of LabView
	\item Begin to build and troubleshoot features in your LabView program
\end{enumerate}

\section*{Textbook Readings}
Note: these readings will help you with your project.  You are responsible for knowing this information before lab.
\begin{itemize}
	\item 3.1: Wires, Cables, and Connectors
	\item 3.5: Resistors
	\item 3.6: Capacitors
	\item 7.3: Multimeters
	\item 7.4: Osciolloscopes
\end{itemize}

\section*{Pre-lab Assignment (to be completed individually)}
Draw a proposed graphical user interface for your project, considering the data values, graphs, and tables you will show. What do you think the user of your device will want to see? What features do you think are most important?

\section*{In-lab Assignment}
Set up a LabView Interface to view and collect the data from your device. How this interface looks and behaves is up to you. You should consider the following when planning your LabView Program:
	\begin{itemize}
		\item How do you want to collect data?
		\item Do you want to view data as a graph or do you want to display numbers?
		\item Do you want to collect data at certain times?
		\item Do you want to user to have a history of the data collected?
	\end{itemize}

The level of complexity will be reflected in your grade, but whatever level of complexity and detail you use, it should work by the time you turn it in. Tip: start slowly and build up from simple to more complex. If you are ready to begin testing, speak to your instructor about the possibility of getting some test samples now.

\end{document}

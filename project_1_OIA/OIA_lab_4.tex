\documentclass{article}

\usepackage{hyperref}
\hypersetup{
	colorlinks=true,
	linkcolor=blue,
	urlcolor=cyan,}
\usepackage{booktabs}
\usepackage{textgreek}

\input{../structure.tex} % Include the file specifying the document structure and custom commands

%----------------------------------------------------------------------------------------
%	ASSIGNMENT INFORMATION
%----------------------------------------------------------------------------------------

\title{OIA Lab 4: Test, Modify, and Calibrate with Real Samples}
\author{BIOE 385 Bioinstrumentation Laboratory} 
\date{}
%----------------------------------------------------------------------------------------

\begin{document}
\large
\maketitle

\section*{Goals}
\begin{enumerate}
	\item Refine your LabView program
	\item Calibrate, test, and modify your program using the nanoshells provided
	\item Consider adding features that would add sophistication or user-friendliness
\end{enumerate}


\section*{In-lab Assignment}
This week you will use actual samples to test in your device. You will be provided with 5 samples of nanoshells of various concentrations. The relative concentrations will be labeled on the cuvettes. You should think about, design and build an appropriate test sample holder for your cuvette. Legos, construction material and a 3D printer are available for you to use. 
\begin{info}
Talk to the instructor about what you need to do to use the 3D printer.	
\end{info}


Test the samples with one of your circuit setups. Are you able to detect differences in the light transmitted through the sample? Are these differences the same as you get on the spectrophotometer for the wavelength of your LED? If things are not working as you desire, think about what could be going wrong. Troubleshoot in VERY small steps.\\

While there are no concrete deadlines and objectives to complete for the lab today, the only reason you would stop working is if everything works perfectly! So, be diligent during this time and you will be relieved of the need to cram and try to finish the lab project outside of lab time.

\end{document}

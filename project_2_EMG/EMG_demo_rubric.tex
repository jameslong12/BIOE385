\documentclass{article}

\input{../structure.tex} % Include the file specifying the document structure and custom commands

%----------------------------------------------------------------------------------------
%	ASSIGNMENT INFORMATION
%----------------------------------------------------------------------------------------

\title{Project 2: EMG Reflex Assignment (EMG)}
\author{BIOE 385 Bioinstrumentation Laboratory} 
\date{Project Demo Rubric}
%----------------------------------------------------------------------------------------

\begin{document}
\large
\maketitle

\textbf{Student Name:}\vspace{0.5cm}

\textbf{Student Name:}\hfill 	\textbf{Total Grade:\ \ \ \ \underline{\hspace{1cm}}/50}\vspace{0.5cm}

\textbf{Student Name:}\hfill\\

User experience:
\begin{itemize}
	\item Ease of use (actions are intuitive, large room for error in operation) \hfill \underline{\hspace{1cm}}/5 \vspace{2cm}
	\item Operation is free of unintentional errors \hfill \underline{\hspace{1cm}}/6\vspace{2cm}
	\item Clear guidelines for safe use \hfill \underline{\hspace{1cm}}/5\vspace{1cm}
\end{itemize}

Presentation:
\begin{itemize}
	\item LabVIEW front panel is well-labeled and organized \hfill \underline{\hspace{1cm}}/5 \vspace{2cm}
	\item Circuit is clean and easy to understand \hfill \underline{\hspace{1cm}}/5 \vspace{2cm}
\end{itemize}


Device accuracy (must be within reasonable physiological range for full credit):
\begin{itemize}
	\item Trial 1 \underline{\hspace{3cm}} \hfill \underline{\hspace{1cm}}/8
	\item Trial 2 \underline{\hspace{3cm}} \hfill \underline{\hspace{1cm}}/8
	\item Trial 3 \underline{\hspace{3cm}} \hfill \underline{\hspace{1cm}}/8
\end{itemize}
\end{document}

\documentclass{article}

\usepackage{hyperref}
\hypersetup{
	colorlinks=true,
	linkcolor=blue,
	urlcolor=cyan,}
\usepackage{booktabs}
%%%%%%%%%%%%%%%%%%%%%%%%%%%%%%%%%%%%%%%%%
% Lachaise Assignment
% Structure Specification File
% Version 1.0 (26/6/2018)
%
% This template originates from:
% http://www.LaTeXTemplates.com
%
% Authors:
% Marion Lachaise & François Févotte
% Vel (vel@LaTeXTemplates.com)
%
% License:
% CC BY-NC-SA 3.0 (http://creativecommons.org/licenses/by-nc-sa/3.0/)
% 
%%%%%%%%%%%%%%%%%%%%%%%%%%%%%%%%%%%%%%%%%

%----------------------------------------------------------------------------------------
%	PACKAGES AND OTHER DOCUMENT CONFIGURATIONS
%----------------------------------------------------------------------------------------

\usepackage{amsmath,amsfonts,stmaryrd,amssymb} % Math packages

\usepackage{enumerate} % Custom item numbers for enumerations

\usepackage[ruled]{algorithm2e} % Algorithms

\usepackage[framemethod=tikz]{mdframed} % Allows defining custom boxed/framed environments

\usepackage{listings} % File listings, with syntax highlighting
\lstset{
	basicstyle=\ttfamily, % Typeset listings in monospace font
}

%----------------------------------------------------------------------------------------
%	DOCUMENT MARGINS
%----------------------------------------------------------------------------------------

\usepackage{geometry} % Required for adjusting page dimensions and margins

\geometry{
	paper=a4paper, % Paper size, change to letterpaper for US letter size
	top=2.5cm, % Top margin
	bottom=3cm, % Bottom margin
	left=2.5cm, % Left margin
	right=2.5cm, % Right margin
	headheight=14pt, % Header height
	footskip=1.5cm, % Space from the bottom margin to the baseline of the footer
	headsep=1.2cm, % Space from the top margin to the baseline of the header
	%showframe, % Uncomment to show how the type block is set on the page
}

%----------------------------------------------------------------------------------------
%	FONTS
%----------------------------------------------------------------------------------------

\usepackage[utf8]{inputenc} % Required for inputting international characters
\usepackage[T1]{fontenc} % Output font encoding for international characters

\usepackage{XCharter} % Use the XCharter fonts

%----------------------------------------------------------------------------------------
%	COMMAND LINE ENVIRONMENT
%----------------------------------------------------------------------------------------

% Usage:
% \begin{commandline}
%	\begin{verbatim}
%		$ ls
%		
%		Applications	Desktop	...
%	\end{verbatim}
% \end{commandline}

\mdfdefinestyle{commandline}{
	leftmargin=10pt,
	rightmargin=10pt,
	innerleftmargin=15pt,
	middlelinecolor=black!50!white,
	middlelinewidth=2pt,
	frametitlerule=false,
	backgroundcolor=black!5!white,
	frametitle={Command Line},
	frametitlefont={\normalfont\sffamily\color{white}\hspace{-1em}},
	frametitlebackgroundcolor=black!50!white,
	nobreak,
}

% Define a custom environment for command-line snapshots
\newenvironment{commandline}{
	\medskip
	\begin{mdframed}[style=commandline]
}{
	\end{mdframed}
	\medskip
}

%----------------------------------------------------------------------------------------
%	FILE CONTENTS ENVIRONMENT
%----------------------------------------------------------------------------------------

% Usage:
% \begin{file}[optional filename, defaults to "File"]
%	File contents, for example, with a listings environment
% \end{file}

\mdfdefinestyle{file}{
	innertopmargin=1.6\baselineskip,
	innerbottommargin=0.8\baselineskip,
	topline=false, bottomline=false,
	leftline=false, rightline=false,
	leftmargin=2cm,
	rightmargin=2cm,
	singleextra={%
		\draw[fill=black!10!white](P)++(0,-1.2em)rectangle(P-|O);
		\node[anchor=north west]
		at(P-|O){\ttfamily\mdfilename};
		%
		\def\l{3em}
		\draw(O-|P)++(-\l,0)--++(\l,\l)--(P)--(P-|O)--(O)--cycle;
		\draw(O-|P)++(-\l,0)--++(0,\l)--++(\l,0);
	},
	nobreak,
}

% Define a custom environment for file contents
\newenvironment{file}[1][File]{ % Set the default filename to "File"
	\medskip
	\newcommand{\mdfilename}{#1}
	\begin{mdframed}[style=file]
}{
	\end{mdframed}
	\medskip
}

%----------------------------------------------------------------------------------------
%	NUMBERED QUESTIONS ENVIRONMENT
%----------------------------------------------------------------------------------------

% Usage:
% \begin{question}[optional title]
%	Question contents
% \end{question}

\mdfdefinestyle{question}{
	innertopmargin=1.2\baselineskip,
	innerbottommargin=0.8\baselineskip,
	roundcorner=5pt,
	nobreak,
	singleextra={%
		\draw(P-|O)node[xshift=1em,anchor=west,fill=white,draw,rounded corners=5pt]{%
		Question \theQuestion\questionTitle};
	},
}

\newcounter{Question} % Stores the current question number that gets iterated with each new question

% Define a custom environment for numbered questions
\newenvironment{question}[1][\unskip]{
	\bigskip
	\stepcounter{Question}
	\newcommand{\questionTitle}{~#1}
	\begin{mdframed}[style=question]
}{
	\end{mdframed}
	\medskip
}

%----------------------------------------------------------------------------------------
%	WARNING TEXT ENVIRONMENT
%----------------------------------------------------------------------------------------

% Usage:
% \begin{warn}[optional title, defaults to "Warning:"]
%	Contents
% \end{warn}

\mdfdefinestyle{warning}{
	topline=false, bottomline=false,
	leftline=false, rightline=false,
	nobreak,
	singleextra={%
		\draw(P-|O)++(-0.5em,0)node(tmp1){};
		\draw(P-|O)++(0.5em,0)node(tmp2){};
		\fill[black,rotate around={45:(P-|O)}](tmp1)rectangle(tmp2);
		\node at(P-|O){\color{white}\scriptsize\bf !};
		\draw[very thick](P-|O)++(0,-1em)--(O);%--(O-|P);
	}
}

% Define a custom environment for warning text
\newenvironment{warn}[1][Warning:]{ % Set the default warning to "Warning:"
	\medskip
	\begin{mdframed}[style=warning]
		\noindent{\textbf{#1}}
}{
	\end{mdframed}
}

%----------------------------------------------------------------------------------------
%	INFORMATION ENVIRONMENT
%----------------------------------------------------------------------------------------

% Usage:
% \begin{info}[optional title, defaults to "Info:"]
% 	contents
% 	\end{info}

\mdfdefinestyle{info}{%
	topline=false, bottomline=false,
	leftline=false, rightline=false,
	nobreak,
	singleextra={%
		\fill[black](P-|O)circle[radius=0.4em];
		\node at(P-|O){\color{white}\scriptsize\bf i};
		\draw[very thick](P-|O)++(0,-0.8em)--(O);%--(O-|P);
	}
}

% Define a custom environment for information
\newenvironment{info}[1][Info:]{ % Set the default title to "Info:"
	\medskip
	\begin{mdframed}[style=info]
		\noindent{\textbf{#1}}
}{
	\end{mdframed}
}
 % Include the file specifying the document structure and custom commands

%----------------------------------------------------------------------------------------
%	ASSIGNMENT INFORMATION
%----------------------------------------------------------------------------------------

\title{BIOE 385: Bioinstrumentation Laboratory}
\author{Fall 2022} 
\date{}
%----------------------------------------------------------------------------------------

\begin{document}

\maketitle
\subsection*{Instructor}
Dr. James Long\\
james.long@rice.edu\\
BRC 765

\subsection*{Teaching Assistants}
\begin{table}[h!]
	\centering
\begin{tabular}[h!]{ccc}
\textit{Monday} & \textit{Wednesday} & \textit{Thursday}\\
Nicole Sevilla & Samira Hajebrahimi & Drew Bonham\\
nicole.sevilla@rice.edu & samira.hajebrahimi@rice.edu & drew.bonham@rice.edu
\end{tabular}
\end{table}

\subsection*{Office Hours}
\textit{BRC 230}: Wednesday and Thursday 4pm-5pm\\
Please email for appointments outside the above times.

\subsection*{Course Website}
The course website, which includes all necessary documents, is hosted on Github. You may access it through \href{https://jameslong12.github.io/BIOE385}{this link} or the following: \href{https://bit.ly/bioe385_f22}{bit.ly/bioe385\_f22}. Assignment submission and grading will be hosted on Canvas.

\subsection*{Course Description}
In this laboratory course, you will learn to design, build, and test biomedical instrumentation using the NI ELVIS hardware system and LabView software. Students will work in teams to complete two design projects: an Optical Immunoassay system and an EMG/Reflex measurement device. Mastery of instrumentation principles and the below course objectives will be evaluated with pre/in-lab assignments, reports, project demos, and written/practical lab exams.

\subsection*{Course Objectives}
By the end of the course, students will demonstrate the ability to:

\begin{enumerate}
	\item Design, build, and troubleshoot electrical circuits to acquire engineering measurements.
	\item Analyze and present acquired data in a user-friendly manner.
	\item Solve open-ended engineering problems, taking resource constraints and design criteria into consideration.
	\item Communicate and justify engineering decisions and designs in a written document.
\end{enumerate}

\subsection*{Prerequisites}
You must have successfully completed ELEC 243 and the associated lab course to complete this class. No exceptions.

\subsection*{Textbook}
Paul Scherz, \underline{Practical Electronics for Inventors}, 4th Ed., McGraw Hill, 2016.\\

This textbook is available for free online as follows:

\begin{enumerate}
	\item If you are accessing from off-campus, \href{https://kb.rice.edu/82263}{log in to the Rice University VPN}
	\item Using OneSearch on the \href{https://library.rice.edu}{Fondren Library Website}, search for "Practical Electronics for Engineers"
	\item In the search results, click the link to McGraw Hill Access Engineering
	\item Double-check that you are browsing the most recent edition (4th ed., pub. 2016)
\end{enumerate}

\subsection*{Safety}
Laboratory environments include inherent dangers and thus strict compliance to safety measures is important. All students are expected to work safely and to ask for assistance when uncertain. Absolutely no eating, drinking or gum chewing will be allowed at any place or time in the lab.

\subsection*{Attendance Policy}
Students are required to attend ALL sessions of the laboratory. Exceptions for illness, family emergencies, or other excusable absences may be granted with instructor approval on a case-by-case basis. Any make-ups for valid absences must be done at the convenience of the instructor and your lab partner.

\subsection*{Grading Policy}

\begin{table}[h!]
	\centering
\begin{tabular}[h!]{cccc}
\toprule
	Category & Submissions & Total Points & \% Final Grade\\
	\midrule
	Pre-Labs/In-Labs & In class & 155 & 31\%\\
	Miscellaneous & In class & 25 & 5\%\\
	Lab Reports & Canvas & 120 & 24\%\\
	Project Demos & In class & 100 & 20\%\\
	Exams & In class & 100 & 20\%\\
	\bottomrule
\end{tabular}
\end{table}

Penalties may be applied for the following violations:
\begin{itemize}
	\item -5 points: Unsafe laboratory practices, leaving lab area messy, or not returning components correctly (per event)
	\item -50 points: Failure to return lab supplies at the end of the semester. A complete list of supplies can be found \href{https://jameslong12.github.io/BIOE385/general_course_materials/supplies.pdf}{here}.
\end{itemize}

For a full list of assignments and due dates, \href{https://jameslong12.github.io/BIOE385/general_course_materials/assignments.pdf}{see here}.

\subsubsection*{Late Policy}
Pre-lab and In-lab assignments will not be accepted late without a valid excused absence. Late reports will be penalized 30\% of the assignment value per day.

\subsubsection*{Regrade Policy}
If you believe that an error was made in grading your assignments, you should write a \underline{short} justification of your claim and attach it to the original homework assignment. You can bring the documents to class or to my office (BRC 765).  Make sure to write your contact email so I can review your concern(s) and respond to you directly. The “statute of limitations” for submitting such claims is one week after the assignment was returned.  I will respond to re-grading requests – but note that the entire report might be re-graded, not just the section in question, in order to ensure consistency and fairness. 

\subsection*{Honor Code Policy}

Collaboration between group members is expected for the completion of high quality lab reports. However, you may not consult the materials from previous years' BIOE 383 or 385 courses, including but not limited to: written reports, posters, data calculations, and code. Additionally, while you may discuss aspects of the lab projects and reports with other groups, your group's submitted work must reflect your individual efforts. In other words, you cannot submit another group's work as your own, and these materials may include but are not limited to: writing, graphs, code, and data calculations. You cannot collaborate on lab exams.

In simpler terms, only submit work that is your own. If you need clarification, please contact the instructor \textit{prior to submission} to avoid an infraction. For a list of standard definitions as outlined by the Honor Council, please see \href{https://cpb-us-e1.wpmucdn.com/blogs.rice.edu/dist/c/490/files/2022/08/Honor-Council-Standard-Definitions-and-Policies.pdf}{this link}.

\subsection*{Commitment to Equitable Learning}
This class is committed to an equitable learning environment. Accommodations will be made for students with alternative needs, but it is critical that you alert the instructor in advance of any additional resources you need prior to assignment submission and lab exams. In particular, you must notify the instructor of any alternative testing needs \textit{at least two weeks prior to an exam.} For additional resources and more information, please visit the \href{https://drc.rice.edu/}{Disability Resource Center} and the \href{https://aop.rice.edu/}{Access and Opportunity portal.}

\subsection*{Schedule}
Please see the \href{https://registrar.rice.edu/calendars/fall-semester-2022}{Rice University Academic Calendar} for important administrative (add/drop/withdraw) dates.\\

{\color{red} Project 1: Optical Immunoassay System (OIA)}\\

{\color{blue} Project 2: EMG Reflex Measurement Device (EMG)}\\

\begin{table}[h!]
	\centering
\begin{tabular}[h!]{cccc}
\toprule
	Week & Lab/Topic & Reading & Pre-Lab/Report\\
	\midrule
	1 (08/22) & {\color{red} OIA 1: Introduction to ELVIS} & 7.1, 2.1-3, 2.9-12, 2.14, 2.16 & \\
	2 (08/29) & {\color{red} OIA 2: Photodetector Circuits} & 5.3, 5.5, 7.2, 2.20, 2.23 & {\color{red} Prelab 2 Due}\\
	\midrule
	3 (09/05) & Labor Day: No lab this week & &\\
	\midrule
	4 (09/12) & {\color{red} OIA 3: LabVIEW Interface} & 3.1, 3.5, 3.6, 7.3, 7.4 & {\color{red} Prelab 3 Due}\\
	5 (09/19) & {\color{red} OIA 4: Finalize Projects} & &\\
	6 (09/26) & {\color{red} OIA 5: \textbf{Project Demos}} & & {\color{red} \textbf{OIA Report Due}}\\
	7 (10/03) & {\color{red} OIA 6: Exam 1} & &\\
	\midrule
	8 (10/10) & Midterm Recess: No lab this week & &\\
	\midrule
	9 (10/17) & {\color{blue} EMG 1: Passive and Active Filters} & 2.31, 7.1-7, 8.1-7, 9 & \\
	10 (10/24) & {\color{blue} EMG 2: Reflex Hammer Circuit} & & {\color{blue} Prelab 2 Due}\\
	11 (10/31) & {\color{blue} EMG 3: Measurement Circuit} & & {\color{blue} Prelab 3 Due}\\
	12 (11/07) & {\color{blue} EMG 4: LabVIEW Interface} & & {\color{blue} Prelab 4 Due}\\
	& & & {\color{blue} \textbf{EMG Report Draft Due}}\\
	13 (11/14) & {\color{blue} EMG 5: Finalize Projects} & & \\
	\midrule
	14 (11/21) & Thanksgiving Recess: No lab this week & & \\
	\midrule
	15 (11/28) & {\color{blue} EMG 6: \textbf{Project Demos}} & & {\color{blue} \textbf{EMG Report Due}}\\
	Finals & {\color{blue} EMG 7: Exam 2} & &\\
	\bottomrule
\end{tabular}
\end{table}


\end{document}
